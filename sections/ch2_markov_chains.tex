\chapter{The Formalism of Markov chains}
In this chapter we will present the formalism of Markov chains, starting from their definition and introducing the mathematical tools to represent them. We will then discuss the asymptotic behaviour of Markov chains and focus on some statistical properties of Markov chains in the asymptotic limit.
\label{ch:markov_chains}
\section{Definitions}
Consider a finite set $\Space$ with dimension $M$ and consider a system whose state can take values in $\Space$; $\Space$ will be called therefore \emph{state space}. Let us indicate the state of the system at time $t$ with the variable $X_t$, and let us furthermore discretize the time, taking $t \in \mathbb{N}$. Suppose now that system state changes randomly in time, so that $X_t$ is a random variable. Let
\begin{equation}
    \prob{X_t = x} \quad , \quad x \in \Space
\end{equation}
be the probability that the system is found in the state labeled by $x$ at time $t$.

\begin{ndef} [Discrete Markov chain] \label{def:markov-chain}
    A sequence of states $\{X_t\}_{t\in \mathbb{N}}$ on a discrete state space $\Space$ is said a \emph{discrete Markov chain} if the following property holds $\forall t \in \mathbb N$ and $\forall x_0, \dots, x_t \in \Space$.
    \begin{equation}
        \prob{ X_t = x_t | X_{t-1} = x_{t-1}, X_{t-2} = x_{t-2}, \dots, X_0 = x_0 } = \prob { X_t = x_t | X_{t-1} = x_{t-1} } \quad
    \end{equation}
The above stated property, usually called \emph{Markov property} but also, more evocatively, \emph{memorylessness}, states that the probability for the system to be in a certain state at a given time instant only depends on the state of the time instant before. This means that the system \enquote{has no memory of the past}: the current state is only affected by the previous state, regardless of the way is was reached.
\end{ndef}

We will now consider a particular kind of Markov chains, called time-homogeneous.
\begin{ndef} [Time-homogeneous Markov chain] A Markov chain is called time-homogeneous if the transition probability between any two given states doesn't depend on time, \ie if
    \begin{equation}
        \prob {X_{t+1} = y | X_{t} = x} = \prob {X_{t} = y | X_{t-1} = x} \quad \quad \forall t\in \mathbb{N}, \forall y, x \in \Space
    \end{equation}
\end{ndef}

As a consequence of time-homogeneity,one is allowed to refer in general to \emph{the} probability of the transition from state $x$ to state $y$: this quantity does not depend on time and is therefore well defined. For the remainder of this chapter we will only be dealing with time-homogeneous Markov chains.

\section{Distributions, stochastic matrices and evolution}
We will now introduce a means of representing the state and the evolution of a system in a Markov process, which is the use of distribution vectors and stochastic matrices. Since the considered chains are \emph{discrete}, it is possible to use a vector to represent the probabilities for the system to be in the possible states at a certain time, and a matrix to represent the probabilities of all the possible transitions between states. Let us now formalize these concepts.

\smallskip
From now on, consider a state space $\Space$ with dimension $M$, and label the possible states with the symbols $x_1, x_2, \dots, x_M$.

\begin{ndef}[Distribution vector] \label{def:distribution}
    A distribution vector, also simply called \emph{distribution}, is a vector $\vec{f} \in \mathbb{R}^M = (f_1, \dots, f_M)$ with the following properties:
    \begin{center}
        \begin{enumerate}
            \item $ f_i \geq 0 \quad \forall i = 1,\dots, M $
            \item $\sum_{i=0}^M f_i = 1$
            \item $f_i = \prob {X = x_i} \quad \forall i = 1,\dots, M$
        \end{enumerate}
    \end{center}
    Notice that this is a good definition, since the sum of the probabilities of all the possible states is 1, and these probabilities are all positive numbers.
\end{ndef}

Distribution vectors contain all the necessary information to completely describe the state of a system. If, for instance, it is certain that the system is in the state $x_k$, then all the components of its distribution will be 0, except for the $k$-th one, which will be 1.

Since we will be dealing with the time evolution of the system, a subscript will be added to the distributions to indicate the time they refer to: $\vec{f}_t$, for instance, will be the distribution of the system at time $t$.

Having seen how to represent the probability distribution of a state, let us now focus on how to represent transition probabilities.

\begin{ndef} [Stochastic matrix] \label{def:stoc_matrix}
    A stochastic (or transition) matrix is a matrix $P \in \mathbb{R}^{M\times M}$ with the following properties:
    \begin{center}
        \begin{enumerate}
            \item $P_{ij} \geq 0 \quad \forall i,j = 1,\dots,M$
            \item $\sum_{i = 0}^M P_{ij} = 1 \quad \forall j = 1,\dots,M$
            \item $P_{ij} = \prob {X_{t+1} = x_i | X_t = x_j } \quad \forall i,j = 1,\dots,M; \forall t\in \mathbb N$
        \end{enumerate}
    \end{center}
\end{ndef}

Notice that while distributions depend on time, stochastic matrices do not, since the considered Markov chains are time-homogeneous and therefore the transition probabilities are time-independent.

\smallskip
Suppose now that a system is described at a certain time by a distribution $\vec{f}_t$. Then, in order to obtain the distribution of the system in the following times instants, one can use the stochastic matrix as shown in the following theorem.

\begin{theorem} \label{th:evolution_simple}
    Let $\f_t$ be the distribution of the system at time $t$. Then, the distribution at time $t + 1$ is given by
    \begin{equation}
        \f_{t+1} = P \f_t
    \end{equation}
\end{theorem}
\begin{proof}
    For all $i = 1,\dots,M$ it holds that
    \begin{align}
        (\f_t)_i
         & = \prob {X_{t+1} = x_i }                                           &  & \text{definition of $\f$}                                     \\
         & = \sum_{j = 0}^M \prob{X_{t+1} = x_i | X_t = x_j} \prob{X_t = x_j} &  & \text{law of total probability}                \\
         & = \sum_{j = 0}^M P_{ij} (\f_t)_j                                   &  & \text{definitions of $P$ and $\f$} \label{eq:matmul1}          \\
         & = (P \f_t)_i                                                       &  & \text{definition of matrix multiplication} \label{eq:matmul2}
    \end{align}

\end{proof}

\begin{corollary} \label{th:evolution_complete}
    Let $\f_0$ be the distribution of the system at time 0. Then, the distribution at time $t$ is given by
    \begin{equation}
        \f_t = P^t \f_0
    \end{equation}
\end{corollary}

\begin{remark}
    It is clear from the proof of Theorem \ref{th:evolution_simple}, and in particular from eq. \ref{eq:matmul1}, that it is useful to interpret distributions as \emph{column vectors}. This follows from the fact that we have defined $P$ such that the sum on its \emph{columns} must always be equal to 1. However, it is also a common practice to define $P$ such that the sum on its \emph{rows} must be equal to 1; after this latter convention, distributions are interpreted as row vectors and the law for the evolution is $\f_{t+1} = \f_t P$. We decided to adhere to the \enquote{column-based} convention because it resembles the formalism of quantum mechanics, where operators are multiplied by the state vectors and not the other way around; this convention results therefore more natural from a physicist's point of view.
\end{remark}

\section{Asymptotic behaviour}
We now draw our attention to a fundamental aspect of Markov chains. Suppose that the initial distribution of a system, $\f_0$, is completely known, \ie that one knows the probabilities for the system to be in each of the possible states $x_1, x_2, \dots, x_M$. We have shown how to compute the evolution of these probabilities for any time step using the stochastic matrix, but we might want to ask ourselves the following question:
\begin{quote}
    \emph{Is there a distribution that the system will eventually reach, asymptotically? If there is, which distribution is it?}
\end{quote}
In other words, this question aims at knowing whether the system can reach \emph{equilibrium}, \ie a configuration in which the probabilities of the possible states are constant in time.
In most real-life applications, indeed, one is usually interested in the properties of a system in conditions of equilibrium, after the system has evolved for a sufficiently long time; hereafter it is useful to understand the asymptotic behaviour of the system. In order tackle this problem we will introduce the concepts of \emph{limiting distribution} and \emph{stationary distribution}.

\begin{ndef}[Limiting distribution]
    Given a Markov chain with stochastic matrix $P$, its limiting distribution $\p$, if it does exist, is defined as the distribution whose components satisfy
    \begin{equation}
        \pi_i \coloneqq \lim_{t \rightarrow \infty} \prob{X_t = x_i} \quad \forall i=1,\dots,M
    \end{equation}
    independently of the initial distribution of the chain.
\end{ndef}
Thus, the $i$-th component of the limiting distribution represents the probability that the system will be in the $i$-th state at equilibrium. An interesting feature of the limiting distribution is that, if it exists, then the matrix obtained by $\lim_{t \rightarrow \infty} P^t$ has each column equal to the limiting distribution:
\begin{equation}
    \text{if $\p$ exists, then } \quad \quad \lim_{t \rightarrow \infty} P^t =
    \begin{pmatrix}
        \pi_1  & \pi_1  & \dots  & \pi_1  \\
        \pi_2  & \pi_2  & \dots  & \pi_2  \\
        \vdots & \vdots & \vdots & \vdots \\
        \pi_M  & \pi_M  & \dots  & \pi_M
    \end{pmatrix}
\end{equation}
This statement is a consequence of Corollary \ref{th:evolution_complete}:
    \begin{align}
        \pi_i
         & = \lim_{t \rightarrow \infty} \prob{X_t = x_i}                                                       \\
         & = \lim_{t \rightarrow \infty} (\f_t)_i                                                               \\
         & = \lim_{t \rightarrow \infty} P^t \f_0         & \text{for corollary \ref{th:evolution_complete}}
    \end{align}   
By definition, $\pi_i$ must not depend on $\f_0$, therefore we can choose as $\f_0$ the vectors of the canonical basis of $\mathbb{R}^M$; each $i$-th element of the canonical basis is then transformed in the $i$-th column of $P^t$, obtaining the above equality.

\medskip
As mentioned before, another important concept that we have to introduce in the context of the asymptotic behaviour is the \emph{stationary distribution}.

\begin{ndef} [Stationary distribution]
    Given a Markov chain with stochastic matrix $P$, its stationary distribution $\staz$, if it does exist, is defined as the distribution that satisfies the follwing equality.
    \begin{equation}
        P \staz = \staz
    \end{equation}
    Hence, the stationary distribution is an \text{eigenvector} of the stochastic matrix with unitary eigenvalue. If the system happens to be found in the stationary distribution, it will remain in that same state forever.
\end{ndef}
The limiting distribution and the stationary distribution are closely related. A first relationship between them is expressed by the following theorem.

\begin{theorem} \label{th:limit-stat}
    Suppose that a Markov chain with stochastic matrix $P$ has a limiting distribution $\p$. Then $\p$ is also a stationary distribution for the chain.
\end{theorem}
\begin{proof}
    Independently of the initial distribution $\f_0$, the following holds.
    \begin{align}
        P \p
         & = P \lim_{t \rightarrow \infty} P^t \f_0  \\
         & = lim_{t \rightarrow \infty} P^{t+1} \f_0 \\
         & = \p
    \end{align}
\end{proof}
The inverse implication, however, does not always hold. That is, a Markov chain might well have a stationary distribution, but $\lim_{t \rightarrow \infty} P^t$ might not exist anyway. First of all, one must assure that such stationary distribution is unique, otherwise the limit would not be unique; even then, however, other assumptions are needed in order to guarantee convergence. The Theorem named after Perron and Frobenius states the sufficient conditions such that the stationary distribution of a chain is unique and coincides with the limiting distribution. In order to present the statement of the Theorem, however, some preliminary definitions are needed. 

\begin{ndef} [Irreducible Markov chain]
    A Markov chain is said to be irreducible if, starting from any state, there exist a number of time steps that allows to reach any other state. That is, 
    \begin{equation}
        \forall x_i, x_j \in \Space \quad \exists t\in \mathbb{N}^+ : \prob{X_t = x_i|X_0 = x_j} > 0
    \end{equation}
    That is, no matter what the initial state for the system is, every other state has a nonzero probability to be reached at some later instant.
\end{ndef}
\begin{ndef} [Period of a state]
    Given a state $x_i \in \Space$, its \emph{period} $k_i$ is defined as the greatest common divisor (gcd) of the time instants at which $x_i$ can be reached, given that the system is in the state $x_i$ at $t=0$. That is,
    \begin{equation}
        k_i \coloneqq \text{gcd} \left\{ t\in \mathbb{N}^+ : \prob{X_t = x_i | X_0 = x_i} > 0 \right\}
    \end{equation}
    
\end{ndef}
\begin{ndef} [Aperiodicity]
    A \emph{state} is said to be \emph{aperiodic} if its period is equal to 1. A Markov chain on space state $\Space$ is said to be aperiodic if all $x_i \in \Space$ are aperiodic. 
\end{ndef}
\begin{ndef} [Ergodicity]
    A Markov chain that is both irreducible and aperiodic is said \emph{ergodic}.
\end{ndef}
Now we are finally equipped to state the Perron-Frobenius Theorem.

\begin{theorem}[Perron-Frobenius] \label{th:perron-frobenius}
    Suppose that a Markov chain is ergodic. Then, it has a unique stationary distribution and a limiting distribution, and the two coincide.
\end{theorem}

\medskip
So we have finally found an answer to our original question: if one can verify that a Markov chain is ergodic, then all one has to do is to find its stationary distribution, \ie the eigenvector with unitary eigenvalue, and then the asymptotic distribution of the system completely known. 

Let us make some considerations about how one can find the stationary distribution in practice. As said above, given the stochastic matrix of a Markov chain, finding its stationary distribution is an eigenvalue problem. Since there exist efficient algorithms to find eigenvectors and eigenvalues of matrices, a possible option is to solve the problem numerically. Alternatively, one might try to find an analytical formula for the stationary distribution; this can be done, for instance, by means of considerations about the behaviour of the system at equilibrium. This approach will be used to explain the stationary distribution of the Ehrenfest chain, in \hyperref[ch:3]{Chapter 3}.


\section{Mean recurrence time}
In this section we will assume that the limiting distribution of the chain is known and we will show some useful quantities that the limiting distribution allows to calculate.

Suppose that a system described by a Markov process is found at $t = 0$ in some state $x_i$. We might be interested in asking: what is the average time that one should wait, before the system will come back to that same state? Alternatively, what is the proportion of time that the system is expected to spend in the state $x_i$?

The knowledge of the asymptotic behaviour of the Markov chain can help answer these questions. As before, we will start with some simple definitions, and then give the relevant theorems.

\medskip
Consider a system involved in a Markov process on state space $\Space$ and suppose that the system evolves for $T$ time steps; let the sequence $(X_0, X_1, \dots, X_T)$ describe the state of the system at the various time instants.
\begin{ndef} [Number of visits]
    Given a state $x_i \in \Space$, we define the random variable $n_i (X_0, T)$ as the number of time instants that system happens to be in state $x_i$, given that the initial state was $X_0$ and that the system evolved for $T$ time steps. This means:
    \begin{equation}
        n_i (X_0, T) \coloneqq \sum_{t = 1}^T \mathds{1}_i(X_t) \quad , \quad \text{given a certain }X_0
    \end{equation}
    where
    \begin{equation}
        \mathds{1}_i(X_t) =
        \begin{cases}
            1 & \text{if $X_t = x_i$} \\
            0 & \text{otherwise}
        \end{cases}
    \end{equation}
\end{ndef}
\begin{ndef} [Mean visit frequency and mean recurrence time] \label{def:nu-tau}
    Given a state $x_i \in \Space$, we define its \emph{mean visit frequency} as
    \begin{equation}
        \nu_i \coloneqq \lim_{T\rightarrow \infty} \frac{1}{T} \mathds{E} \left[ n_i (T) \right]
    \end{equation}
    independently on the initial state --- for this reason we have dropped the indication of the initial state.
    We furthermore define the \emph{mean recurrence time} of state $x_i$ as
    \begin{equation} \label{def:mean_rec_time}
        \tau_i \coloneqq \frac{1}{\nu_i}
    \end{equation}
\end{ndef}
Notice that the definition of the mean recurrence time can be interpreted as the \emph{average} time one has to wait before the system can return to a given state.

\medskip
There is a a fundamental relationship between the mean visit frequency and the limiting distribution of the chain, which is described by the following theorem.

\begin{theorem}
    Suppose that a Markov chain has a limiting distribution $\p$. Then the mean visit frequency of each state is given by the corresponding component of the limiting distribution:
    \begin{equation}
        \nu_i = \pi_i
    \end{equation}
\end{theorem}
\begin{proof}
   Suppose that the system starts at $X_0 = x_j$; then 
    \begin{equation} \label{eq:expected_value}
        \mathds{E} [ n_i (X_0 = x_j, T)] = \sum_{t = 1}^T \prob{X_t = x_i | X_0 = x_j} = \sum_{t = 1}^T P_{ij}^t
    \end{equation} 
    If the chain has a limiting distribution $\p$, then after Theorem \ref{th:limit-stat} $\p$ is also a stationary distribution for the matrix $P$. This implies that, for each $t\in \mathbb{N}$
    \begin{equation} \label{eq:property_pi}
        \p = P^t \p  \quad \text{or, for the components}\quad \pi_i = \sum_{j = 1}^M P_{ij}^t \pi_{j}
    \end{equation}
    Then
        \begin{align}
            \pi_i 
                &  = \frac{1}{T} \sum_{t = 1}^T \pi_i && \forall T\\
                &  = \lim_{T \rightarrow \infty} \frac{1}{T} \sum_{t = 1}^T \pi_i \\
                & = \lim_{T \rightarrow \infty}  \frac{1}{T} \sum_{t = 1}^T \left(\sum_{j = 1}^M P_{ij}^t \pi_{j}\right) &&  \text{Eq. \ref{eq:property_pi}}\\
                & = \sum_{j = 1}^M \pi_j \left(\lim_{T \rightarrow \infty}  \frac{1}{T}  \sum_{t = 1}^T P_{ij}^t \right) && \text{changing the order of finite sums} \\
                & = \sum_{j = 1}^M \pi_j \nu_i && \text{\hyperref[def:nu-tau]{definition of $\nu$}} \\
                & = \nu_i && \text{\hyperref[def:distribution] {Property 2  of the distributions}}
        \end{align}
\end{proof}