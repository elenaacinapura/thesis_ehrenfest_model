\chapter{The Formalism of Markov Chains}
\label{ch:markov_chains}
\section{Definitions}
Consider a finite set $\Space$ with dimension $M$ and consider a system whose state can take values in $\Space$; $\Space$ will therefore be called \emph{state space}. Let us indicate the state of the system at time $t$ with the variable $X_t$, and let us furthermore discretize the time, taking $t \in \mathbb{N}$. Suppose now that system state changes randomly in time, so that $X_t$ is a random variable. Let
\begin{equation}
    \prob{X_t = x} \quad , \quad x \in \Space
\end{equation}
be the probability that the system is found the state labeled by $x$ at time $t$.

\begin{ndef} [Discrete Markov chain] \label{def:markov-chain}
    A sequence of states $\{X_0, X_1, X_2, \dots, X_t\}$ on a discrete state space $\Space$ is said a \emph{discrete Markov chain} if the following property holds $\forall t \in \mathbb N$ and $\forall x_0, \dots, x_t \in \Space$:
    \begin{equation}
        \prob{ X_t = x_t | X_{t-1} = x_{t-1}, X_{t-2} = x_{t-2}, \dots, X_0 = x_0 } = \prob { X_t = x_t | X_{t-1} = x_{t-1} } \quad
    \end{equation}
\end{ndef}

The above stated property, usually called \emph{Markov property} but also, more evocatively, \emph{memorylessness}, basically states that the probability for the system to be in a certain state at a give time only depends on the state of the time before. This means that the system "has no memory of the past": only the current state can affect the future, and not the way the system has reached it.

\smallskip
We will now consider a particular kind of Markov chains, namely \emph{time-homogeneous} ones.
\begin{ndef} [Time-homogeneous Markov chain] A Markov chain is called time-homogeneous if the transition probability to some certain state to another doesn't depend on time, i.e. if
    \begin{equation}
        \prob {X_{t+1} = y | X_{t} = x} = \prob {X_{t} = y | X_{t-1} = x} \quad \quad \forall t\in \mathbb{N}, \forall y, x \in \Space
    \end{equation}
\end{ndef}

As a consequence of time-homogeneity, we are allowed to refer in general to \emph{the} probability of the transition from state $x$ to state $y$, since this quantity does not depend on time and is therefore well defined. For the remainder of this thesis we will only be dealing with time-homogeneous Markov chains.

\section{Distributions, stochastic matrices and evolution}
Having defined the basic properties of a Markov chain, we will now introduce a powerful means of representing the state and the evolution of a system in a Markov process, which is the use of distribution vectors and stochastic matrices. Since we are considering \emph{discrete} chains, the number of the possible states if finite, and thus the idea is to use a vector to represent the probabilities for the system to be in the possible states at a certain time, and a matrix to represent the probabilities of all the possible transitions between states. Let us now formalize these concepts.

\smallskip
Consider a state space $\Space$ with dimention $M$, and label the possible states with the symbols $x_1, x_2, \dots, x_M$.

\begin{ndef}[Distribution vector] \label{def:distribution}
    A distribution vector, also simply called \emph{distribution}, is a vector $\vec{f} \in \mathbb{R}^M$ with the following properties:
    \begin{center}
        \begin{enumerate}
            \item $f_i \geq 0 \quad \forall i = 1,\dots, M$
            \item $\sum_{i=0}^M f_i = 1$
            \item $f_i = \prob {X = x_i} \quad \forall i = 1,\dots, M$
        \end{enumerate}
    \end{center}
    Notice that this is a good definition, since the sum of the probabilities of all the possible states is 1, as it should be, and these probabilities are all positive numbers.
\end{ndef}

Distribution vectors contain all the necessary information to completely describe the state of a system. If, for instance, it is certain that the system is in the state $x_k$, then all the components of its distribution will be 0, except for the $k$-th one, which will be 1. In general, however, the components of a distribution will be nonzero.
Furthermore, since we will be dealing with the time evolution of the system, we will add a subscript to the distributions to indicate the time that the distribution refers to: $\vec{f}_t$, for instance, will be the distribution of the system at time $t$.

\smallskip
Having seen how to represent the probability distribution of a state, let us now concentrate on how to represent transition probabilities.

\begin{ndef} [Stochastic matrix] \label{def:stoc_matrix}
    A stochastic (or transition) matrix is a matrix $P \in \mathbb{R}^{M\times M}$ with the following properties:
    \begin{center}
        \begin{enumerate}
            \item $P_{ij} \geq 0 \quad \forall i,j = 1,\dots,M$
            \item $\sum_{i = 0}^M P_{ij} = 1 \quad \forall j = 0,\dots,M$
            \item $P_{ij} = \prob {X_{t+1} = x_i | X_t = x_j } \quad \forall i,j = 1,\dots,M; \forall t\in \mathbb N$
        \end{enumerate}
    \end{center}
\end{ndef}

Notice that while distributions depend on time, stochastic matrices do not, since we are considering time-homogeneous Markov chains, and therefore the transition probabilities are time-independent.

\smallskip
Suppose now that a system is described at a certain time by a distribution $\vec{f}_t$. Then, in order to obtain the distribution of the system in the following times, one can use the stochastic matrix thanks to the rule of conditional probabilities, as it is shown in the following theorem.

\begin{theorem} \label{th:evolution_simple}
    Let $\f_t$ be the distribution of the system at time $t$. Then, the distribution at time $t + 1$ given by
    \begin{equation}
        \f_{t+1} = P \f_t
    \end{equation}
\end{theorem}
\begin{proof}
    For all $i = 1,\dots,M$ it holds that
    \begin{align}
        (\f_t)_i
         & = \prob {X_{t+1} = x_i }                                           &  & \text{for the def. of $\f$}                                     \\
         & = \sum_{j = 0}^M \prob{X_{t+1} = x_i | X_t = x_j} \prob{X_t = x_j} &  & \text{for the rule of conditional probabilities}                \\
         & = \sum_{j = 0}^M P_{ij} (\f_t)_j                                   &  & \text{for the def. of $P$ and $\f$} \label{eq:matmul1}          \\
         & = (P \f_t)_i                                                       &  & \text{for the def. of matrix multiplication} \label{eq:matmul2}
    \end{align}

\end{proof}

\begin{corollary} \label{th:evolution_complete}
    Let $\f_0$ be the initial distribution of the system, at time 0. Then, the distribution at time $t$ given by
    \begin{equation}
        \f_t = P^t \f_0
    \end{equation}
\end{corollary}

\begin{remark}
    It is clear from the proof of theorem \ref{th:evolution_simple}, in particular from eq. \ref{eq:matmul1}, that it is useful to interpret distributions as \emph{column vectors}. This, indeed, follows from the fact that we have defined $P$ such that the sum on its \emph{columns} must always equal 1. However, it is also a common practice to define $P$ such that the sum on it \emph{rows} must equal 1; after this latter convention, then, distributions are interpreted as row vectors and one obtains that the law fot the evolution is $\f_{t+1} = \f_t P$. We decided to adhere to the "column-based" convention because it resembles the formalism of quantum mechanics, where operators are multiplied by the state vectors, and not the other way around; this convention results therefore more natural from a physicist's point of view.
\end{remark}

\section{Limiting behaviour}
We now draw our attention to a fundamental aspect of Markov chains. Suppose that the initial distribution of a system, $\f_0$, is completely known, i.e. that we know the probabilities for the system to be in each of the possible states $x_1, x_2, \dots, x_M$. We know how to compute che evolution of these probabilities time step after time step using the stochastic matrix, but we might want to ask ourself the following question:
\begin{quote}
    \emph{Is there a distribution that the system will eventually reach, asymptotically? If it does, which distribution is this?}
\end{quote}
In other words, this question aims at knowing whether the system can reach \emph{equilibrium}, i.e. a configuration in which the probabilities of the possible states are constant in time.
It should be clear why one should be interested in this question: in most real-life applications, one is usually interested in the properties of a system in conditions of equilibrium, after the system has evolved for a sufficiently long time. It is therefore useful to understand the asymptotic (or \emph{limiting}) behaviour of the system, and in order tackle this problem we will introduce some concepts known as \emph{limiting distribution} and \emph{stationary distribution}.

\begin{ndef}[Limiting distribution]
    Given a Markov chain with stochastic matrix $P$, its limiting distribution $\p$, if it does exist, is defined as the distribution whose components are satisfy
    \begin{equation}
        \pi_i \equiv \lim_{t \rightarrow \infty} \prob{X_t = x_i} \quad \forall i=1,\dots,M
    \end{equation}
    independently of the initial distribution of the chain.
\end{ndef}
Thus, the components of the limiting distribution give the probabilities to find the system in a given state asymptotically, when the system will have reached equilibrium. An interesting feature of the limiting distribution is that, if it exists, then the matrix obtained by $\lim_{t \rightarrow \infty} P^t$ has each column equal to the limiting distribution:
\begin{equation}
    \text{if $\p$ exists, then } \quad \quad \lim_{t \rightarrow \infty} P^t =
    \begin{pmatrix}
        \pi_1  & \pi_1  & \dots  & \pi_1  \\
        \pi_2  & \pi_2  & \dots  & \pi_2  \\
        \vdots & \vdots & \vdots & \vdots \\
        \pi_M  & \pi_M  & \dots  & \pi_M
    \end{pmatrix}
\end{equation}
This is quite straightforward indeed, thanks to corollary \ref{th:evolution_complete}:
\begin{equation}
    \begin{aligned}
        \pi_i
         & = \lim_{t \rightarrow \infty} \prob{X_t = x_i}                                                       \\
         & = \lim_{t \rightarrow \infty} (\f_t)_i                                                               \\
         & = \lim_{t \rightarrow \infty} P^t \f_0         & \text{for corollary \ref{th:evolution_complete}}
    \end{aligned}   
\end{equation}
Since, by definition, the $\pi_i$ must not depend on $\f_0$, then we can choose as $\f_0$ the vectors of the canonical basis of $\mathbb{R}^M$; each $i$-th element of the canonical basis is then transformed in the $i$-th column of $P^t$, and thus we obtain the above equality.

\medskip
As mentioned before, another important concept that we have to introduce in the context of limiting behaviour is the \emph{stationary distribution}.

\begin{ndef} [Stationary distribution]
    Given a Markov chain with stochastic matrix $P$, its stationary distribution $\staz$, if it does exist, is defined as the distribution such that
    \begin{equation}
        P \staz = \staz
    \end{equation}
    Hence, the stationary distribution is the \text{eigenvector} of the stochastic matrix with unitary eigenvalue. If the system happens to be found in the stationary distribution, it will remain in that same state forever.
\end{ndef}
As one might guess, the limiting distribution and the stationary distribution are closely related. A first relationship between them is expressed by the following theorem.

\begin{theorem} \label{th:limit-stat}
    Suppose that a Markov chain with stochastic matrix $P$ has a limiting distribution $\p$. Then $\p$ is also a stationary distribution for the chain.
\end{theorem}
\begin{proof}
    We have that, independently of the initial distribution $\f_0$:
    \begin{align}
        P \p
         & = P \lim_{t \rightarrow \infty} P^t \f_0  \\
         & = lim_{t \rightarrow \infty} P^{t+1} \f_0 \\
         & = \p
    \end{align}
\end{proof}
Unfortunately, the other implication does not always hold. That is, a Markov chain might well have a stationary distribution, but $\lim_{t \rightarrow \infty} P^t$ may not exist anyway. First of all, one must assure that such stationary distribution is unique, otherwise the limit would not be unique; even then, however, other assumptions are needed in order to guarantee convergence. Fortunately, the theorem named after Perron and Frobenius states when the stationary distribution of a chain is also its limiting distribution, but before we proceed with its statement we must give some more definitions. We will not prove the theorem, since that would be a lengthy task that would draw our attention away from the physical applications of the Markov chains, so we will also try to give an intuitive and practical definition of the following properties, rather than a strictly formal one.
% Agggiungere reference a dove trovare i dettagli dei calcoli

\begin{ndef} [Irreducible Markov chain]
    A Markov chain is said to be irreducible if every state can lead, sooner or later, to every other state. That is, no matter what the initial state for the system is at $t=0$, every other state has a nonzero probability to be reached at some time $t>0$.
\end{ndef}
\begin{ndef} [Period of a state and aperiodicity]
    Given a state $x_i \in \Space$, its \emph{period} $k_i$ is defined as the greatest common divisor (gcd) of the times at which $x_i$ can be reached, given that the system started in $x_i$ at $t=0$. That is,
    \begin{equation}
        k_i \equiv \text{gcd} \left\{ t>0 : \prob{X_t = x_i | X_0 = x_i} > 0 \right\}
    \end{equation}
    
\end{ndef}
\begin{ndef} [Aperiodicity]
    A \emph{state} is said to be \emph{aperiodic} if its period equals 1. A Markov chain on space state $\Space$ is said to be aperiodic if all $x_i \in \Space$ are aperiodic. 
\end{ndef}
\begin{ndef} [Ergodicity]
    A Markov chain that is both irreducible and aperiodic is said \emph{ergodic}.
\end{ndef}
Now we are finally equipped to state the Perron-Frobenius theorem.

\begin{theorem}[Perron-Frobenius]
    Suppose that a Markov chain is ergodic. Then it has a unique stationary distribution and a limiting distribution, and the two coincide.
\end{theorem}

\medskip
So we have finally found an answer to our original question: if we can verify that a Markov chain is ergodic, then all we have to do is to find its stationary distribution (i.e., the eigenvector with unitary eigenvalue), and then asymptotic distribution of the system is automatically known. Finding eigenvectors and eigenvalues is usually a very feasible task and there are plenty of libraries that offer optimized routines to perform that.

In the last section of this chapter, then, we will show some interesting information that one can obtain about a Markov chain, thanks to the knowledge of the limiting distribution.

\section{Mean recurrence time}
Suppose that a system described by a Markov process is found at $t = 0$ in some state $x_i$. An interesting question at this point could be: what is the average time that one should have to wait, before the system will come back to that same state? Alternatively, a closely related question might be: what is the proportion of time that the system is expected to spend in state $x_i$?

As one might expect, the knowledge of the limiting behaviour of the Markov chain can help answer these questions, and this section aims at doing that. As before, we will start with some simple definitions, and then give the relevant theorems.

\medskip
Consider a system involved in a Markov process on state space $\Space$ and suppose that the system evolves for $T$ time steps; let the sequence $\{X_0, X_1, \dots, X_T\}$ describe the state of the system at the various time instants.
\begin{ndef} [Number of visits]
    Given a state $x_i \in \Space$, we define $n_i (X_0, T)$ as the number of times that system happens to be in state $x_i$, given that the initial state was $X_0$ and that the system evolved for $T$ time steps. This means:
    \begin{equation}
        n_i (X_0, T) \equiv \sum_{t = 1}^T \mathbb{1}_i(X_t) \quad , \quad \text{given a certain }X_0
    \end{equation}
    where
    \begin{equation}
        \mathbb{1}_i(X_t) =
        \begin{cases}
            1 & \text{if $X_t = x_i$} \\
            0 & \text{otherwise}
        \end{cases}
    \end{equation}
\end{ndef}
\begin{ndef} [Mean visit frequency and mean recurrence time] \label{def:nu-tau}
    Given a state $x_i \in \Space$, we define its \emph{mean visit frequency} as
    \begin{equation}
        \nu_i \equiv \lim_{T\rightarrow \infty} \frac{1}{T} \mathbbm{E} \left[ n_i (T) \right]
    \end{equation}
    independently on the initial state -- for this reason why have dropped the indication of the the initial state.
    We furthermore define the \emph{mean recurrence time} to state $x_i$ as
    \begin{equation}
        \tau_i \equiv \frac{1}{v_i}
    \end{equation}
\end{ndef}
Notice that the definition of the mean recurrence time can be interpreted as the \emph{average} time one has to wait before the state can return to a given state.

\medskip
There is a a fundamental relationship connecting the mean visit frequency and the mean recurrence time to the limiting distribution of the chain.

\begin{theorem}
    Suppose that a Markov chain has a limiting distribution $\p$. Then the mean visit frequency of each state is given by the corresponding component of the limiting distribution:
    \begin{equation}
        \nu_i = \pi_i
    \end{equation}
\end{theorem}
\begin{proof}
   Suppose that the system starts at $X_0 = x_j$; then 
    \begin{equation} \label{eq:expected_value}
        \mathbbm{E} [ n_i (X_0 = x_j, T)] = \sum_{t = 1}^T \prob{X_t = x_i | X_0 = x_j} = \sum_{t = 1}^T P_{ij}^t
    \end{equation} 
    Now, if the chain has a limiting distribution $\p$, then after theorem \ref{th:limit-stat} $\p$ is also a stationary distribution for the matrix $P$. This implies that, $\forall t$
    \begin{equation} \label{eq:property_pi}
        \p = P^t \p  \quad \text{or, for the components}\quad \pi_i = \sum_{j = 1}^M P_{ij}^t \pi_{j}
    \end{equation}
    Then
    \begin{equation}
        \begin{aligned}
            \pi_i 
                &  = \frac{1}{T} \sum_{t = 1}^T \pi_i && \forall t\text{, then also}\\
                &  = \lim_{T \rightarrow \infty} \frac{1}{T} \sum_{t = 1}^T \pi_i \\
                & = \lim_{T \rightarrow \infty}  \frac{1}{T} \sum_{t = 1}^T \left(\sum_{j = 1}^M P_{ij}^t \pi_{j}\right) &&  \text{for eq. \ref{eq:property_pi}}\\
                & = \sum_{j = 1}^M \pi_j \left(\lim_{T \rightarrow \infty}  \frac{1}{T}  \sum_{t = 1}^T P_{ij}^t \right) && \text{changing the order of finite sums} \\
                & = \sum_{j = 1}^M \pi_j \nu_i && \text{by definition of \hyperref[def:nu-tau]{$\nu$}} \\
                & = \nu_i && \text{for property 2 of the \hyperref[def:distribution] {distributions}}
        \end{aligned}
    \end{equation}
\end{proof}