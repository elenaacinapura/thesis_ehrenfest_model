\chapter{Conclusions}
The study of the process of diffusion by means of the Ehrenfest model has helped to understand in detail the concept of irreversibility. The theory of Markov chains, in particular, results to be particularly suited to the study of this topic because it enables to calculate meaningful properties of a system at equilibrium. The calculation of the mean recurrence time, specifically, allows to quantify irreversibility and give a more precise meaning to the concept of the \enquote{arrow of time}.

The additional study of the problem through a simulation has confirmed the predictions of the Markov theory and has furthermore strengthened the understanding of the evolution of the system towards the asymptotic state. In addition, the attempt to estimate the asymptotic properties of the system has led to make considerations about the speed of convergence and to grasp with hand the de-facto irreversibility of a process.

The analytical tools offered by the theory of Markov chains extend far beyond the ones presented in this dissertation. The limiting distribution and the mean recurrence time are in fact only some of the quantities that can be calculated in the Markov theory. Probably, the most interesting quantity that is worth mentioning is entropy: by adopting Shannon's definition of entropy, it can be calculated very straightforwardly and complete the understanding of irreversible processes. The calculation of entropy would surely be the most natural continuation of this dissertation.