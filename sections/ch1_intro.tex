\chapter{Introduction}
One of the main issues in the history of thermodynamics has been the apparent conflict between irreversibility and recurrence. A lot of thermodynamic processes, such as diffusions and phase transitions, cannot be reversed without external intervention, although Newtonian mechanics allows a spontaneous recurrence to the initial state. It seems that some processes can only happen in a certain time order, but the evolution of systems obeying to classical mechanics should be time-reversible. Among the various attempts to shed some light on this problem, two approaches deserve some attention. On one hand, one can adopt a statistical approach and introduce the concept of probability into the description of physical processes; this is the approach that eventually gave rise to the field of statistical mechanics. On the other hand, the improvements of computers' performances have opened the possibility of simulating the dynamics of thermodynamic processes directly at a microscopic level.

This dissertation investigates the presented problem by means of a model proposed by Ehrenfest in the early years of the 20\textsuperscript{th} century for the process of the diffusion of a gas. According to this model, the diffusion can be described as a stochastic process in which particles are randomly selected and moved between parts of the available volume. This approach is different from the ones usually adopted in the basic theories of statistical mechanics, and it is mostly useful if it is combined with a particular theory, which is the theory of \emph{Markov chains}.

Markov chains are a mathematical model for describing stochastic processes that present a particular property, called \emph{Markov property}. This property is satisfied by a vast variety processes, and for this reason Markov chains have a wide range of applications in fields beyond thermodynamics, including economics, signal processing and information theory. For the purposes of thermodynamics, they represent a useful instrument because they allow to study with great detail the properties of systems at \emph{equilibrium}, which is fundamental for understanding the concept of reversibility. For instance, Markov chains allow to find the probability for a process to be reversed, or the average amount of time needed to see a certain state when the system is at equilibrium. For this reason, Markov chains will be the main mathematical tool to study the Ehrenfest problem in this dissertation.

As mentioned at the beginning, another option to study thermodynamic processes is to use a computational approach. A simulation of the Ehrenfest model, in addition to giving more insight into the evolution of the system towards equilibrium, also provides a way to test and strengthen the prediction of the Markov theory.

The dissertation is organized in three chapters. The first chapter introduces the formalism of Markov chains, with a particular focus on their properties in the asymptotic limit. The second chapter presents the Ehrenfest model and analyses it by means of Markov chains. Finally, the last chapter describes a simulation of the Ehrenfest model and presents its results, comparing them with the predictions of the Markov theory. 