\chapter*{Abstract}
\addcontentsline{toc}{chapter}{Abstract}
The diffusion of a gas is one of the most simple and at the same time important thermodynamic processes for understanding the mechanisms of irreversibility. This dissertation investigates these mechanisms by studying the \emph{Ehrenfest urn model}, which describes the diffusion of a gas as a stochastic process. This model is first analysed by means of the mathematical theory of Markov chains, which is used in particular to predict the behaviour and characteristics of the system at equilibrium. The model is then studied with a computational approach by simulating the Ehrenfest problem, and the results are finally compared with the ones of the Markov theory. The combination of the analytical and computational methods eventually results in a good understanding of the concept of irreversibility, both from a qualitative and from a quantitative point of view.