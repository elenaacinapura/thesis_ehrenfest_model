\chapter{The Ehrenfest Chain}
In the previous chapter we introduced the mathematical formalism of Markov chains and investigated in particular some properties of Markov chains in the asymptotic limit, in a condition of equilibrium. In this chapter we will then apply this knowledge to study a specific problem, namely the \emph{Ehrenfest urn problem}. The Ehrenfest urn problem is a \emph{Gedankenesperiment} formulated in an attempt to study the statistical properties of a gas diffusing in a volume. We will first introduce the problem, and then see how it can be very naturally formalized by means of Markov chains.
%spiegare perche questo modello ci interessa
\section{The Ehrenfest Urn Problem}
Consider a system of $N$ particles distributed in 2 boxes, which are going to be called \emph{box 1} and \emph{box 2}. Suppose that, a regular time intervals, you select a particle at random and a box at random, and you put the selected particle in the selected box. The number of particles in each box will change in time, and we are interested in the configuration of the particles after a sufficiently long period of time - that is, at equilibrium.

The just illustrated situation is based on very few and simple assumption, but is nevertheless very useful to describe real processes of diffusion. For instance, we can use it to model an experiment in which a gas is initially spread out in a certain volume; suddenly a wall is removed, giving access to another equal portion of space, and particles start diffusing the in whole space available. We can, at least in principle, keep track of how many particles are still the first sector, and how many are instead in the other one. We might then be interested in asking ourself questions such as: what is the probability that all the particles will all come back together to the original portion of volume? How much time does the system spend in a given configuration with $k$ particles on one side and $N - k$ on the other? These are very deep questions that involve concepts such as entropy and they are central in understanding in which sense it is true that thermodynamical processes are subject to "the arrow of time", even if Newtonian mechanics is not. 

If we rest on the assumption that we can treat diffusion problems by modeling them as a stochastic process obeying some rules as those formulated by Ehrenfest, then we can find precise answers to the mentioned questions thanks to Markov chains.
