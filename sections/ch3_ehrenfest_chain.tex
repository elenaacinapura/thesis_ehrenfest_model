\chapter{The Ehrenfest Chain}
In the previous chapter we have introduced the mathematical formalism of Markov chains and have investigated in particular the condition of asymptotic equilibrium. In this chapter we will then apply this knowledge to study a specific problem, namely the \emph{Ehrenfest urn problem}. The Ehrenfest urn problem is a \emph{Gedankenesperiment} formulated in an attempt to study the statistical properties of a gas diffusing in a volume. We will first introduce the problem, and then see how it can be very naturally formalized by means of Markov chains.

\section{The Ehrenfest urn problem}
Consider a system of $N$ particles distributed in 2 boxes, which are going to be called \emph{box A} and \emph{box B}. Suppose that, a regular time intervals, you select a particle at random and a box at random, and you put the selected particle in the selected box. The number of particles in each box will change in time, and we are interested in properties of the system after a sufficiently long period of time - that is, at equilibrium.

The just illustrated situation consists in very few and simple rules, but is nevertheless very useful to describe real processes of diffusion. For instance, we can use it to model an experiment in which a gas is initially spread in a certain volume; suddenly a wall is removed, giving access to another equal portion of space, and particles start diffusing the in whole space available. We can, at least in principle, keep track of how many particles are still the first sector, and how many are instead in the other one. We might then be interested in asking ourselves questions such as: what is the probability that the particles will all come back together to the original portion of volume? How much time does the system spend on average in a given configuration with $k$ particles on one side and $N - k$ on the other? These are by no means trivial questions, and they are central in understanding a fundamental question that arises when dealing with thermodynamic processes: why is there in thermodynamics an "arrow of time" that is otherwise absent is the principles of newtonian mechanics?

If we rest on the assumption that we can treat diffusion problems by modeling them as a stochastic processes obeying some rules as those formulated by Ehrenfest, then we can find precise answers to the mentioned questions thanks to Markov chains.

\section{Formalization through Markov chains}
Let us now translate the Ehrenfest urn problem into the language of Markov Chains. 
First of all, notice that the state of the system can be completely described by giving the number of particles in one of the two boxes; let us chose box A. Let then $X_t$ represent the number of particles in box A at time $t$ -- clearly, the number of particles in box B is then given by $N - X_t$. $X_t$ can take values $0, 1, 2, \dots, N$, for a total of $N + 1$ possible states. So to summarize
\begin{equation}
    X_t \in \Space = \{ 0, 1, 2, \dots, N \}
\end{equation}
In the previous chapter the possible states of the system were labeled by $x_1, x_2, \dots, x_M$; in this section will will turn to a more compact and easy-to-read notation: if there are $i$ particles in box A, we will just label that state with the letter $i$:
\begin{equation}
    \forall i = \{0, 1, \dots, N \} \quad X_t = i \quad \Leftrightarrow \quad \text{there are $i$ particles in box A at time $t$}
\end{equation}
We can now see that a sequence $\{X_t\}_{t\in \mathbb{N}}$ does in fact satisfy the Markov property, thus representing a Markov chain.